\subsection{Domain of Application}\label{subsec:domain-of-application}
The chosen domain of application for this recommender system is restaurants.

One place the restaurant domain differs from a domain involving a selection of products is that of item accessibility.
It is assumed that any recommendable product, if chosen, will be readily available - if a product is not available in an area, either the user is not recommended it, recommended an alternative, or provided with an alternative means of access.

Conversely, willingness to go to a restaurant is far more dependent on location, particularly when recommendations are desired - if someone wanted to travel a non-trivial distance, it is most likely they already know what they want.
Given this, it is proposed the recommender system take into account the proximity of users to restaurants.

\subsection{Review of Related Work}\label{subsec:related-work-review}
~\cite{sawant2013yelp} presents a hybrid recommender using the Yelp dataset with content-based and collaborative filtering.
It implements, compares and evaluates several learning algorithms when applied to find both user and item profiles.

~\cite{mcdpk2016} demonstrates a non-hybrid CF recommender which uses the Yelp dataset.
It uses matrix factorisation to learn a set of latent factors contributing to rating scores.

\subsection{Motivations and Aim}\label{subsec:aim}
~\cite{chernev2015choice} has shown that an increased variety of available products tend to negatively affect sale conversion rates.
A key reason is that being presented with lots of options to pick from can be overwhelming, especially when the user does not know exactly what they want.
Hence, while many restaurants of differing varieties are available, it is better to present users with a sub-set of these.

Recommender systems aim to determine which subset of these products should be shown to the user, looking for the highest likelihood of the user purchasing them.
This rationale also applies to restaurants - especially in an age of deliveries and takeaways due to the pandemic.
The ultimate goal is to provide relevant personalised recommendations to known-users.
