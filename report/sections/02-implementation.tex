\subsection{Input Interface}\label{subsec:input-interface}
The input command-line interface (CLI) is a state-based system.
A pre-set list of commands with specified arguments are used for data input, which in turn updates the internal state.
For example, when a user runs the~\texttt{userid <id>} command, the user data corresponding to the provided ID is fetched and stored.
In addition to being shown on start, a~\texttt{privacy} command is provided to give the user information about the data and how it's being used.
The CBF and CF recommenders have their models trained on load, so it is not possible to update user profiles.

\subsection{Recommendation Algorithm}\label{subsec:recommendation-algorithm}
The system is provided with the selected user ID and desired recommendation count ($n$).
Previously reviewed restaurants are identified and calls down to the CF and CBF recommenders to retrieve the $n$ most similar candidates per reviewed restaurant.
These are pooled together and have their similarities normalised.
They are sorted by similarity and the $n$ most similar candidates are returned as the actual recommendations.

The CBF recommender uses the \texttt{TfidfVectorizer} to pre-calculate phrase features and combines a resultant item-item similarity matrix with the similarity matrix from the other features.
For rating predictions, CBF finds the $k$ nearest items to that being rated and returns them.

The CF recommender uses a sparse user-item matrix and generates an item-item cosine similarity matrix to provide fast kNN lookup.
CF uses a similar algorithm to CBF for rating prediction except that it uses $k$NN on user similarity.

\subsection{Output Interface}\label{subsec:output-interface}
Restaurant recommendations are provided in a user-friendly table-like format and consist of the recommendation number (rank), restaurant name and location, score, predicted star rating and whether the restaurant offers delivery or takeaway.
As part of the Covid-19 adjustments there is an option for displaying restaurants with takeaway or delivery services.

During loading, RAM usage spikes between 7--9 gigabytes, decreasing when temporary calculation data is freed up.
